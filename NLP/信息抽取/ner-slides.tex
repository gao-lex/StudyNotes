\documentclass[9pt,aspectratio=169]{ctexbeamer}

%\usepackage[UTF8,noindent]{ctexcap}
\usepackage{beamerfoils}%logoon、logooff
\usepackage{graphicx}
\usepackage{amsmath}
\usepackage{makecell}
\usepackage{multirow}
\usefonttheme[onlymath]{serif}
\usetheme{Antibes}%Copenhagen Antibes  Berlin  AnnArbor
\usepackage{mathptmx}% 字体
\usepackage{helvet}% 字体
\usepackage{listings}
%\usebeamer
%\usepackage{tikz}
%\usepackage{beamerfoils}
%\usepackage{pgf}
%\MyLogo{
%	\pgfputat{\pgfxy(0,8)}{\pgfbox[right,base]{\includegraphics[scale=0.1]{../../assets/images/eb20.png}}}
%}
\lstset{ % General setup for the package
	language=Python,
	numbers=left,
	frame=shadowbox,
	tabsize=4,
	breaklines=true, 
}
\usecolortheme{beaver}
\newtheorem{thm}{定理}

\title{“达观杯”文本智能信息抽取挑战赛——命名实体识别}
\author{刘同存\\王闯\\高磊}
%\institute{北京邮电大学\\东信北邮}
\logo{\includegraphics[scale=0.08]{../../assets/images/bupt.jpeg}}
\date{\today}

\begin{document}
	
	\begin{frame}
		\titlepage
	\end{frame}
	\section{赛题背景}
	\begin{frame}
		信息抽取(information extraction),即从自然语言文本中,抽取出特定的事件或事实信息,帮助我们将海量内容自动分类、提取和重构。文本智能抽取是信息检索、智能问答、智能对话等人工智能应用的重要基础,它可以克服自然语言非形式化、不确定性等问题,发掘并捕获其中蕴含的有价值信息,进而用于业务咨询、决策支持、精准营销等方面,对产业界有着重要的实用意义。
		
		\begin{figure}
			\centering
			
			\includegraphics[width=\linewidth]{../../assets/images/daguan-question-bg.jpg}
		\end{figure}
	\end{frame}

	\section{赛题理解}
%	\subsection{重要性}
	\begin{frame}
		
		\begin{enumerate}
			\item 重要性:命名实体识别是多种自然语言处理技术的重要基础,对于句法分析、语法分析、语义分析等都有着极其重要的影响,主要应用在信息抽取、机器翻译、问答系统等方面;
			\item 比赛任务:本次大赛的任务是给定一定数量的标注语料以及海量的未标注语料,在$a,b,c$ 3个字段上做信息抽取任务;
			\item 评价指标:信息抽取的评估指标是$F1$值,是正确率和召回率的调和平均值。\begin{gather*}
			\text{正确率} = \frac{\text{抽取出的正确字段数}}{\text{抽取出的字段数}} \\
			\text{召回率} = \frac{\text{抽取出的正确字段数}}{\text{样本的字段数}}\\
			F1 = \frac{2 * \text{正确率}*\text{召回率}}{\text{正确率}+\text{召回率}}
			\end{gather*}
		\end{enumerate}
		
	\end{frame}


	\section{数据说明}
	
	\begin{frame}
		\begin{itemize}
			\item corpus.txt:大规模未标注语料,提供给参赛者用于训练语言模型;
			\item train.txt:此数据集用于训练模型,每一行对应一条文本数据;
			\item test.txt:此数据集用于测试。数据格式同train\_set.txt一致,但不包含标注;
			\item submit\_sample.txt:此文件是预测结果文件的一个样例,和test\_set.txt中每行一一对应。
		\end{itemize}
		\begin{table}[]
			\begin{tabular}{|c|l|}
				\hline
				语料文件               & 样例                                                                         \\
				\hline
				corpus.txt         & \makecell[l]{4509\_20808\_8197\_17159\_17441\_17145\_5908\\\_7706\_5470\_632\_9391\_15274} \\
				\hline
				train.txt          & \makecell[l]{7212\_17592\_21182/c  8487\_8217\_14790/a \\ 19215\_4216/o}                    \\
				\hline
				test.txt           & 2289\_11123\_11122\_17773\_13487\_5323                              \\
				\hline
				submit\_sample.txt & 2289/c 11123\_11122/b 17773\_13487\_5323/o         \\
				\hline
				                
			\end{tabular}
		\end{table}
	\end{frame}
	
	\section{解题思路}
	
	\begin{frame}
		基础方法:
		\begin{itemize}
			\item 监督学习,依赖于需要特征工程的监督学习算法:CRF
			\item 深度学习,利用神经网络提取特征:BiLSTM-CRF
		\end{itemize}
%		\newline
		提升方法:
		\begin{itemize}
			\item 预训练词向量:word2vec/glove
			\item 预训练+微调:Bert
		\end{itemize}
		其他尝试:
		\begin{itemize}
			\item 不同的optimizer:Adma→rAdma
			\item attention机制
			\item 数据增强
			\item 标注方式:BIO、BIOES、BIOS
			\item 调参
		\end{itemize}
		
		
		
		
		
	\end{frame}
	
	\section{条件随机场}
	\subsection{定义}
%\LogoOff
	\begin{frame}
		
%		\frametitle{条件随机场CRF}
%		\framesubtitle{定义}
		
		条件随机场(conditional random field)是给定随机变量$X$条件下,随机变量$Y$的马尔可夫随机场\footnote{马尔科夫随机场,又称概率无向图模型,是一个可以由无向图表示的联合概率分布}。
		
		在序列标注问题\footnote{所谓序列标注问题是指,对于一个一维线性输入序列$\mathbf{x} = \{x_{1},x_{2},\cdots,x_{n}\}$,给线性序列中的每个元素打上标签集合中的某个标签:$\mathbf{y}=\{y_{1},y_{2},\cdots,y_{n}\}$}中主要应用了定义在线性链上的特殊的条件随机场,称为线性链条件随机场。
		
		\begin{thm}
			设在给定随机变量序列$X$的条件下,随机变量序列$Y$的条件概率分布$P(Y|X)$构成条件随机场,即满足马尔科夫性\begin{gather}
				P(Y_{i}|X,Y_{1},\cdots,Y_{i-1},\cdots,Y_{i+1},\cdots,Y_{n}) = P(Y_{i}|X,Y_{i-1},Y_{i+1}) \notag \\
				i=1,2,\cdots,n(\text{在}i=1\text{和}n\text{时只考虑单边})
			\end{gather}则称$P(Y|X)$为线性链条件随机场。
			
		\end{thm}
		
%		$X=\{X_{1},X_{2},\cdots,X_{n}\}$,$Y=\{Y_{1},Y_{2},\cdots,Y_{n}\}$均为线性链表示的随机变量序列,若在给定
		
%		定理11.4
%		定义$G=(V,E)$为一个无向图,$Y={Y_{v}|v}$玛,即V中的每个节点对应于一
%		个随机变量所表示的标记序列的元素YV。如果每个随机变量Y对于G遵守马尔
%		可夫属性,
%		而且在给定
%		即前面所提到的条件独立性,那么(X,  Y)就构成一个条件随机场,
%		X和所有其他随机变量Y{uluxv,{u,二、}的条件下,随机变量Yv的概率
			
		
		
%		\begin{equation}
%			P(\mathbf{Y}_{v}|\mathbf{X},\mathbf{Y}_{w},w\ne v) = P(\mathbf{Y}_{v}|\mathbf{X},\mathbf{Y}_{w},w\sim v)
%		\end{equation}
		
		
	\end{frame}
%\LogoOn
	\subsection{参数化形式}
	\begin{frame}

	

		\begin{figure}
			\centering
			\includegraphics[width=0.3\linewidth]{../../assets/images/CRF-struct2.jpg}
			\caption{链状条件随机场的无向图结构}
		\end{figure}
	
		链状条件随机场假设在各个输出节点之间存在一阶马尔可夫独立性,而并不对$X$做任何的独立假设。
		
		设$P(Y|X)$为线性条件随机场,则在随机变量$X$取值为$x$的条件下,随机变量$Y$取值为$y$的条件概率有以下形式\footnote{Hammersley-Clifford定理:描述了概率无向图的联合概率分布}:
		\begin{gather}
			P(y|x) = \frac{1}{Z(x)}\exp\left(\sum_{i,k}\lambda_{k}t_{k}\left(y_{i-1},y_{i},x,i\right)+\sum_{i,l}\mu_ls_l\left(y_{i},x,i\right)\right) \\
			Z({x}) = \sum_y\exp\left(\sum_{i,k}\lambda_{k}t_{k}\left(y_{i-1},y_{i},x,i\right)+\sum_{i,l}\mu_ls_l\left(y_{i},x,i\right)\right)
		\end{gather}
	\end{frame}
	
	\begin{frame}
		\begin{figure}
			\centering
			\includegraphics[width=0.3\linewidth]{../../assets/images/CRF-struct2.jpg}
			\caption{链状条件随机场的无向图结构}
		\end{figure}
		\begin{gather*}
		P(y|x) = \frac{1}{Z(x)}\exp\left(\sum_{i,k}\lambda_{k}t_{k}\left(y_{i-1},y_{i},x,i\right)+\sum_{i,l}\mu_ls_l\left(y_{i},x,i\right)\right) \\
		Z({x}) = \sum_y\exp\left(\sum_{i,k}\lambda_{k}t_{k}\left(y_{i-1},y_{i},x,i\right)+\sum_{i,l}\mu_ls_l\left(y_{i},x,i\right)\right)
		\end{gather*}
		
		$ t_{k} $是定义在边上的特征函数(共有$ k $个),称为转移特征。表示对于观察序列$ X $,其标注序列在$ i $及$ i-1 $位置上标记的转移概率;
		
		$ s_{l} $是定义在节点上的特征函数(共有$ l $个),称为状态特征。表示对于观察序列$ X $及其$ i $位置的标记的概率。
		
		通常,特征函数$t_{k}$和$s_{l}$取值为1或0;当满足特征条件时取值为1,否则为0。
		条件随机场完全由特征函数$t_{k}$,$s_{l}$和对应的权值$\lambda_{k}$,$\mu_{l}$确定。
	\end{frame}


	\subsection{训练}
	
	
	
	

	\subsection{特征}
	\begin{frame}[fragile]
		把特征函数$ t_{k}\left(y_{i-1},y_{i},x,i\right) $和$ s_l\left(y_{i},x,i\right)\right) $可以抽象为一个特征函数$ f_{m}(y_{i-1},y_{i},x,i),(m=k+l)$。
		
		可通过特征模板来构造特征函数:
\begin{lstlisting}
# Unigram
U00:%x[-2,0]
U01:%x[-1,0]
U02:%x[0,0]
U03:%x[1,0]
U04:%x[2,0]
U05:%x[-2,0]/%x[-1,0]/%x[0,0]
U06:%x[-1,0]/%x[0,0]/%x[1,0]
U07:%x[0,0]/%x[1,0]/%x[2,0]
U08:%x[-1,0]/%x[0,0]
U09:%x[0,0]/%x[1,0]

# Bigram
B
\end{lstlisting}
		
一般选取大小为2的观察窗口为,即$(w_{-2},w_{-1},w_{0},w_{1},w_{2})$。理论上来说,窗口越大,可利用的上下文信息越多,但窗口开得过大除了会严重降低运行效率,还会产生过拟合现象;


		
	\end{frame}
	
	\subsection{CRF工具包——CRF++}
	\begin{frame}
		通过CRF++来实现命名实体识别:
		
		\begin{enumerate}
			\item 转换训练集格式为CRF++所需:\begin{table}
				\begin{tabular}{cc}
				\multirow{6}*{7212\_17592\_21182/c 8487\_8217\_14790/a 19215\_4216/o  $\Rightarrow$  }            & 7212 B-c   \\
				& 17592 I-c \\
				& 21182 I-c              \\
				& 8487 B-a \\
				& 8217 I-a \\
				& $\cdots$ \\
			\end{tabular}
			\end{table}
			\item crf\_learn 学习参数值
			\item crf\_test 对验证集进行预测
			\item 依照所需提交文件格式还原测试集(1的逆运算)
		\end{enumerate}
	\end{frame}

	\subsection{总结}
	\begin{frame}
		CRF的总结:
		优点:
		\begin{itemize}
			\item 训练和预测的时间、空间代价小
			\item 利用了上下文语言环境
		\end{itemize}
		缺点:
		\begin{itemize}
			\item 依然需要手动构造特征
			\item 这个比赛的数据集中,只有观测序列和状态序列两项,没有词性等其他特征辅助构造特征函数
		\end{itemize}
		可以使用深度学习的方法来解决这些缺点
	\end{frame}
	
%
%	\begin{frame}{条件随机场CRF}{特征}
%		内容...
%	\end{frame}

	\section{总结}
	\begin{frame}
		共1258支队伍,名次45
		\begin{table}[]
			\begin{tabular}{cc}
				\hline
				方法               & 成绩   \\
				\hline
				CRF         & 86 \\
%				\hline
				BiLSTM-CRF         & 89               \\
%				\hline
				Bert-BiLSTM-CRF          & 92                              \\
				\hline
				
			\end{tabular}
		\end{table}
	\end{frame}
\end{document}